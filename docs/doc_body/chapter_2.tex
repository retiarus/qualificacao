\chapter{IONOSFERA}

A ionosfera é uma região ionizada da alta atmosfera, estendendo-se entre 60Km até 10000Km de altitude, assim, engloba partes da mesosfera, termosfera, e exosfera. Esta camada constitui-se de íons, elétrons livres criados primariamente por processo de fotoionização, e alguma porção de gás neutro. A fotoionização ionosférica consiste de um processo físico-químico, onde alguma espécies químicas presentes na atmosfera ganham ou perdem elétrons decorrentes da absorção de radiação solar predominantemente nas faixas mais altas do ultravioleta e raios-X \cite{RISNBETH:1969, NEGRETI:2012}. A ionização, também, pode ocorrer devido a colisões com partículas altamente energéticas, provindas do meio solar ou galácticas, são mais facilmente observadas em altas latitudes, e na região da Anomalia Magnética do Atlântico Sul.

A ionosfera pode ser divida em regiões, faixas de altitudes, as quais se diferenciam pelos processos físicos e químicos que governam o comportamento daquela faixa, além disso, este mesmos processos podem variar devido a quantidade de radiação solar recebida, logo, vai-se observar diferenças entre a noite e o dia. Ao longo da noite, a camada F é a única que apresenta uma ionização significativa, enquanto as camadas E e D apresentam um valor extremamente baixo de ionização. Durante o dia, a camada D e E se tornam mais ionizadas, assim como a camada F, que se divide em duas regiões, F1 que é mais fracamente ionizada, e F2 que é mais intensamente ionizada. A camada F2 existe durante a noite e durante o dia, sendo a principal responsável pela reflexão e refração dos sinais de rádio.

A camada D é a mais interna, estando entre 60Km à 90Km acima da superfície da Terra. Sua ionização é devido a radiação do hidrogênio ionizado na série de Lyman-alpha no comprimento de onda de 121.6nm ionizando o óxido nítrico, $NO$, presente na camada. Além disso, raios X altamente energéticos, com comprimento de onda inferior de 1nm podem ionizar as moléculas de $N_2$ e de $O_2$. A camada D apresenta alta taxa de recombinação, de modo que existem mais moléculas neutras do que íons. Tem uma taxa de absorção considerável para ondas de radio de média e alta baixa frequências, e baixas frequências apresentam elevada atenuação, principalmente, devido a absorção de energia pelo elétrons livres, o que aumenta as chances de colisão. Este efeito desaparece durante à noite, devido a uma menor ionização. Pode apresentar valores elevados de ionização em altas latitudes em decorrência de erupções solares com grandes quantidades de matéria hadrônica, prótons, em sua maioria, com uma duração de 24 à 48 horas.

A camada E é a intermediária e está situada entre 90Km à 150Km acima da superfície da Terra. A ionização decorre principalmente devido ao espalhamento de raio-X leve (entre 1 e 10nm) e ultravioleta distante (UV) provindos do Sol com moléculas de oxigênio. A estrutura vertical da camada E é determinada em sua maior parte pela competição entre efeitos de ionização e de recombinação. É importante pela presença de correntes elétricas que nela fluem e interagem com o campo magnético \cite{KIRCHHOFF:1991}. A noite, a camada E quase desaparece, pois sua fonte primaria de ionização não está presente.

A camada F se estende de 150Km à mais de 500Km acima da superfície da Terra. Apresenta a maior concentração de elétrons, portanto, sinais que são capazes de penetrar até esta são capazes de escapar para o espaço. Predominam, nesta, a ionização de átomos de oxigênio por meio de radiação solar no espectro do extremo ultravioleta, entre, 10nm e 100nm. A camada é subdivida em duas regiões a F2 que está presente durante o dia e a noite, e a F1 que aparece somente durante o dia. 

A subcamada F2 se inicia aproximadamente a 300Km de altitude, englobando toda a região superior da ionosfera, inclusive a região de pico da densidade de elétrons. Este máximo no perfil vertical de ionização decorre do balanço entre os processos de transporte de plasma e os processos físico-químicos. Acima deste pico, a ionosfera se encontra em equilíbrio difusivo, ou seja, o plasma se distribui com a sua própria escala de altura. A presença do campo magnético contribui para a distribuição da ionização.

\section{Anomalias na ionosfera}

A ionosfera apresenta várias anomalias, ou seja, várias irregularidades na distribuição de elétrons. Este trabalho tem interesse na anomalia equatorial. Esta aparece aproximadamente entre 15 e 20 graus de latitude magnética, tanto no hemisfério norte, quanto no hemisfério sul, na camada F2. Consiste na formação de uma região de alta densidade eletrônica, e é uma anomalia, pois a densidade de plasma deveria ser maior em regiões equatoriais, e não em latitudes magnéticas mais altas.

Sua origem decorre da deriva vertical do plasma da camada F na região equatorial: o processo de ionização da camada F faz surgir um campo elétrico, apontando para leste, enquanto o campo magnético aponta para o norte, considerando então $\vec{E}\times\vec{B}$, tem-se o surgimento de uma força perpendicular ao campo magnético e ao campo elétrico, o que neste caso, aponta para cima, deslocando o plasmas para regiões de mais alta altitudes. Agora, quando em altas altitudes, o plasma for efeito gravitacional e diferença de pressão é trazido de volta à altitudes mais baixas, porém este movimento de descida é mais eficiente ao longo das linhas de campo magnético, levando a um aumento na densidade de plasma em regiões de médias latitudes. 

A distribuição do plasma também pode ser alterada pela ação de outras variáveis, como o vento. São José dos Campos, encontra-se na região da anomalia equatorial, mais especificamente no pico da anomalia, ou seja, na região onde densidade de plasma em altas altitudes atinge seu valor máximo.

\section{TEC, VTEC e S4}
TEC, ou conteúdo total de elétrons é uma quantidade descritiva utilizada para avaliar a densidade do plasma ionosférico. É o número total de elétrons integrado entre dois pontos, ao longo de de um tubo com seção de 1 m$^2$, ou seja, é uma densidade numérica de elétrons, ao longo, de uma coluna. Geralmente é reportada em unidades de TEC (TECU), definido com $TECU=10^{16}$ el/m$^2$. É importante para determinar a cintilação e os atrasos de fase e de grupo em ondas de rádio no meio.

VTEC, ou conteúdo total de elétrons vertical é projeção do TEC, ao longo de uma linha normal a superfície da Terra, em outras palavras, ela fornece o conteúdo de elétrons ao longo da normal para cada ponto da superfície da Terra.

S4 corresponde ao desvio padrão da intensidade do sinal de GPS, e é utilizado para medir cintilação ionosférica, que é a perda do sinal. Este fenômeno tem sua origem em não homogeneidades na distribuição do plasma ionosférico. Dentro das diversas possíveis irregularidades, que levam a ocorrência de cintilação, existe um interesse particular nas bolhas de plasma.

\section{Bolhas de Plasma}

As bolhas ionosféricas podem ser definidas como regiões de baixa densidade de plasma ionosférico quando comparadas com a sua vizinhança. Utilizando medidas de VTEC é possível definir essa diferença como 30-50 TECU \cite{TAKAHASHI:2006}.

São originadas na região equatorial, após a rápida elevação do plasma, devido a anomalia equatorial, isto é, o plasma ao acender cria regiões de baixa densidade. Após sua formação podem evoluir para altas altitudes (centenas de quilômetros), estendendo-se ao longo das linhas de campo magnético (milhares de quilômetros) nas direções norte-sul, alcançado em torno de 20 graus de latitude magnética.
